\documentclass{article}
\usepackage[utf8]{inputenc}
\usepackage[french]{babel}
\frenchbsetup{StandardLists=true} % à inclure si on utilise \usepackage[french]{babel}
\usepackage{enumitem}
\usepackage{amssymb}
\usepackage{xcolor}

\title{Le Robot NAO en tant qu’assistant de salle d’attente}
\author{B. Déchamps, G. Dekeyser, L. Dumont, G. Desforges, T. Viel}
\date{January 2018}

\begin{document}

\maketitle

\textbf{Mots clés :} Robotique, 
\section{Introduction}

\paragraph{} Le but de ce projet est de programmer le robot NAO afin qu'il permette de faciliter la gestion des patients en salle d’attente. Cela peut être chez un médecin (spécialiste/généraliste) ou dans une maison de retraite. Il s'avère particulièrement utile pour les personnes âgées, pour les personnes à mobilité réduite, pour les personnes ayant des problèmes de vue par exemple.
Son rôle secondaire est de simplifier le travail du médecin et/ou des secrétaires.


\section{Objectifs}
Notre objectif pour ce projet est donc de rendre le robot NAO capable d'effectuer les tâches suivantes, en interaction avec les patients:
\begin{itemize}
    \item Accueillir de nouveaux arrivants dans la salle d'attente, associer leur visage à leur nom, les informer des dernières informations sur leur rendez-vous, les aider à s'asseoir et leur proposer de la lecture.
    \item Gérer les passages: prévenir les patients lorsque le médecin est prêt à les recevoir, en associant les horaires de passage préfixés avec le retard (ou l'avance) potentiel du médecin sur son emploi du temps
    \item Indiquer aux patients le lieu de consultation, et une fois ceux-ci sortis s'assurer qu'ils trouvent la sortie. Oublier le visage une fois que le patient a quitté le cabinet.
\end{itemize}
\paragraph{}\underline{Limites prévues du projets:} Nos objectifs sont bien sur soumis aux limitations techniques du robot NAO, à savoir: \\
-Ses déplacements sont lents et peu stables en présence d'obstacles \\
-De façon générale son comportement peu difficilement être prévu de A à Z en situation réelle, surtout lorsqu'il est en interaction avec un public non initié.


\section{Implémentation}

\subsection*{Fonctionnement général}
Le robot NAO est connecté à une base de données \texttt{SQL} contenant les noms de patients attendus, leur horaire de rendez-vous et quelques informations complémentaires si celles-ci sont utiles.
Il associera au moment de l'accueil le visage du patient à son nom (via la même base de données) pour pouvoir interagir plus facilement et donner accès au client aux données connues par le robot.

\subsection*{Accueil du client}

\noindent L'accueil du client par le robot se déroule de la façon suivante : 

\begin{itemize}[parsep=0.1cm,itemsep=0.1cm,topsep=0.1cm]
    \item Le robot NAO dit bonjour au patient et regarde si le visage est dans la base de données
    \begin{itemize}[label=\rightarrow, font=\small, leftmargin=0.5cm ,parsep=0cm,itemsep=0cm,topsep=0cm]
        \item Boucle de détection et reconnaissance de visage
    \end{itemize}
    \item Il lui demande son prénom et son nom
    \item Il vérifie qu'il est dans la base donnée. 
    
    \begin{itemize}[label=\rightarrow, font=\small, leftmargin=0.5cm ,parsep=0cm, itemsep=0cm, topsep=0cm]
        \item S'il n'y est pas, il redemande le prénom
    \end{itemize}
    \item Il associe ensuite le prénom à un visage s'il n'a pas été reconnu précédemment 
    
    \begin{itemize}[label=\rightarrow, font=\small, leftmargin=0.5cm ,parsep=0cm,itemsep=0cm,topsep=0cm]
        \item Boucle de mémorisation de visage
    \end{itemize}
    
    \item Il propose à l'utilisateur de s'asseoir
    
    
\end{itemize}

Le robot a accès aux données par une \texttt{REST API} développée avec \texttt{Flask} en \texttt{Python}. On utilise la librairie \texttt{httplib} dans l'interface Chorégraphe pour accéder à cette API.


\subsection*{Gestion dans la salle d'attente}
À n'importe quel moment (sous réserve que l'attention du robot ne soit pas accaparée par un autre patient), le patient déjà enregistré peu demander à NAO les informations auxquelles celui-ci a accès: horaire du rdv, temps d'attente estimé etc. NAO reconnaîtra son visage ou en cas de problème lui redemandera son nom pour savoir comment interroger la base de données. \\
De plus, lorsque l'un des patient peut être pris en charge (à l'heure de son rendez-vous ou plus tard si retard du médecin), NAO appelle le patient en question pour lui indiquer que son tour est arriver et lui montrer la salle de la consultation.

\subsection{Langages & Librairies}
voici la liste des langages, librairies et interfaces utilisées dans notre projet:
\begin{itemize}
    \item Le code source est écrit en Python
    \item Choregraphe pour l'interfaçage entre le code Python et le robot NAO
    \item La base de données utilisée est codée en language SQL
    \item Le serveur utilisé pour accueillir la base SQL est codé en Flask
\end{itemize}

\end{document}
